\maketableofcontents

\intro

В современном мире, стремительное развитие технологий оказывает значительное влияние на повседневную жизнь человека. Одной из ключевых областей, где технологии играют важную роль, является доступ к информации и культурным ресурсам. Библиотеки, как центры знаний и культурного наследия, остаются важным элементом образовательной и интеллектуальной среды. Однако традиционная модель функционирования библиотек сталкивается с рядом вызовов: быстрый рост цифровых ресурсов, изменение пользовательских предпочтений, а также необходимость оптимизации процесса обслуживания и предоставления доступа к ним.

Одной из современных тенденций является популяризация аренды книг через цифровые платформы, что предоставляет удобство и упрощает доступ к библиотечным ресурсам. Это особенно актуально в условиях городской жизни, где время и доступность ресурсов играют важную роль. Разработка веб-сайтов и мобильных приложений для аренды книг может значительно улучшить взаимодействие пользователей с библиотеками, увеличивая их посещаемость и привлекая новые категории пользователей.

Примером успешного применения подобных технологий является проект цифровой платформы «Московская электронная библиотека», которая предоставляет доступ к сотням тысяч книг, журналов и других материалов. Пользователи могут брать книги в аренду онлайн и читать их как в электронном, так и в традиционном формате. Это показывает, что библиотеки не теряют своей актуальности, а, напротив, приспосабливаются к новым условиям и сохраняют свою роль в обществе.

Актуальность данной темы обусловлена также изменением культурных и потребительских привычек пользователей, которые все больше предпочитают онлайн-сервисы для удовлетворения своих потребностей. В условиях пандемии COVID-19 был зафиксирован резкий рост интереса к удаленным библиотечным услугам, что подчеркнуло необходимость развития и совершенствования цифровых решений для библиотек.

Данный курсовой проект нацелен на разработку веб-сайта аренды книг в библиотеках города, который будет решать задачи оптимизации работы библиотек, упрощения доступа к книгам. В рамках проекта будет создан интерфейс, удобный для поиска и аренды книг, а также для управления личными данными пользователей и их библиотечными активностями.

Целью работы является разработка системы бронирования книг в библиотеках города. Для ее достижения необходимо выполнить следующие задачи.

\begin{enumerate}
  \item Произвести анализ аналогичных решений.
  \item Проанализировать предметную область.
  \item Спроектировать архитектуру распределенной системы.
  \item Произвести выбор стека технологий для реализации.
  \item Реализовать распределенную систему аренды книг.
\end{enumerate}
