\conclusion

В рамках данной курсовой работы была проведена разработка веб-приложения для аренды книг в библиотеках города, включающая как клиентскую, так и серверную части. На основании проведённого анализа современных технологий были выбраны оптимальные инструменты для создания высокопроизводительного и масштабируемого решения. В качестве бэкенд-фреймворка был использован FastAPI языка программирования Python, обеспечивающий высокую производительность, поддержку асинхронного программирования и простоту интеграции с современными инструментами. Для фронтенда был выбран React, отличающийся гибкостью, большим сообществом и поддержкой современных подходов к созданию интерактивных пользовательских интерфейсов.

Разработанное приложение предоставляет удобный интерфейс для поиска и аренды книг, а также обеспечивает автоматизированный сбор и обработку данных о пользователях и аренде. В процессе реализации были решены задачи по организации архитектуры приложения, обеспечению безопасности данных.

Данная работа продемонстрировала возможность применения современных технологий для создания удобных и функциональных веб-приложений, способных решать задачи цифровизации библиотек и улучшать доступ к знаниям для широкого круга пользователей.

